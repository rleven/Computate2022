\newpage
\section{Aufgabe 1: Linear kongruente Generatoren}
\label{sec:auf1}

\subsection{a)}
Siehe \textit{aufgabe1.py}.

\subsection{b)}

In \autoref{fig:b1} sind zwei Histogramme übereinander abgebildet.
Das orangene Histogramm entspricht der Verteilung der Zufallszahlen nach der in der Aufgabenstellung a) gestellten Formel.\\
Das blaue Histogramm ist dieselbe Verteilung, jedoch mit dem MT19937 berechnet.

\begin{figure}[H]
    \centering
    \includegraphics[scale=0.7]{plots/a1b1.pdf}
    \caption{Histogramm generiert aus den Startwerten aus Aufgabenteil (i)}
    \label{fig:b1}
\end{figure}

In \autoref{fig:b2} werden die Histogramme erneut dargestellt, jedoch diesmal mit den Startwerten aus Aufgabenteil (ii).

\begin{figure}[H]
    \centering
    \includegraphics[scale=0.7]{plots/a1b2.pdf}
    \caption{Histogramm generiert aus den Startwerten aus Aufgabenteil (ii)}
    \label{fig:b2}
\end{figure}

Es ist deutlich erkennbar, dass die Histogramme in \autoref{fig:b2} nahezu gleiche Verteilungswerte aufweisen, wohingegen in \autoref{fig:b1} klare Kanten sichtbar sind.