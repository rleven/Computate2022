\newpage
\section{Aufgabe 2: Beliebige Verteilungen}
\label{sec:auf2}

\subsection{a)}
    Hier sollte die Box-Müller-Transformation implementiert werden, mit der aus zwei gleichverteilten Zufallsvariablen eine zweidimensionale Standardnormalverteilung
    \begin{equation}
        f(x_1, x_2)  = \frac{1}{\sqrt{2 \pi}} e^{\frac{-(x_1^2 + x_2^2)^2}{2}}
    \end{equation}
    erzeugt werden kann.
    Die Transformation sieht aus wie folgt:
    \begin{align}
        x_1 &= \sqrt{-2 \log(u_1)} \cos(2 \pi u_2) \\
        x_2 &= \sqrt{-2 \log(u_1)} \sin(2 \pi u_2) \\
    \end{align}
    \begin{figure}[H]
        \centering
        \includegraphics[width=1\textwidth]{plots/BoxMueller.png} \vspace*{-0.5cm}
        \caption{Hier ist die mit der Box-Müller-Methode bestimmte und analytische 2d-Standardnormalverteilung in 100 Bins aufgetragen.}
        \label{fig:BoxMueller}
    \end{figure}
    \FloatBarrier

    Es ist die Funktion
    gegeben.
    Zuerst wurde die Methode \verb|determineSupportingPoints()| implementiert, um die drei für die Intervallhalbierung nötigen Stützstellen grob zu bestimmen.

    Danach wird die Methode \verb|bisection()| für die beiden Anfangsbedingungen, bei einer Abbruchbedingung von $\epsilon = 10^{-7}$, durchgeführt und man erhält die lokalen Minima
    \begin{equation}
        \vec{x}_{\mathrm{min}} \approx 
        \begin{pmatrix}
            0 \\ 0
        \end{pmatrix} \; , \qquad
        \vec{x}_{\mathrm{min}} \approx 
        \begin{pmatrix}
            0 \\ 1
        \end{pmatrix} \; .
    \end{equation}
    


\subsection{b)}
    Der zentrale Grenzwertsatz besagt, dass wenn die Summe $S_n = \sum_{i=1}^n u_i$ von $n$ gleichverteilten Zufallsvariablen $u_i$ gebildet und daraus die Variable
    \begin{equation}
        Z_n = \frac{S_n - n\mu}{\sigma \sqrt{n}}
    \end{equation}
    erzeugt wird, die Wahrscheinlichkeitsdichte von $\lim_{n \rightarrow \infty} Z_n$ gegen die Standardnormalverteilung mit $\mu=0$ und $\sigma=1$ geht.
    
    \begin{figure}[H]
        \centering
        \includegraphics[width=0.7\textwidth]{plots/zentralerGrenzwertsatz.png} \vspace*{-0.5cm}
        \caption{Hier ist die mit dem zentralen Grenzwertsatz bestimmte und analytische Standardnormalverteilung in 50 Bins aufgetragen.}
        \label{fig:zentralerGrenzwertsatz}
    \end{figure}
    \FloatBarrier

    Ein Nachteil könnte bei dieser Methode sein, dass man bei einer endlichen kleineren Anzahl an gleichverteilten Zufallsvariablen eine Gaußverteilung bekommt, deren Mittelwert und Standardabweichung nicht bekannt ist.

    Aus diesem Grund wurde eine analytische Gaußverteilung mit den Parametern
    \begin{align*}
        \mu \approx 1,731 \\
        \sigma \approx 0,292
    \end{align*}
    an die berechnete Verteilung gefittet.


\subsection{d)}
    Es soll die Transformation gefunden werden, die die Cauchy-Verteilung
    \begin{equation}
        p(x) = \frac{1}{\pi} \frac{1}{1 + x^2}
    \end{equation}
    mit $x \in (-\infty, \infty)$ aus einer Gleichverteilung $f(u)=\mathrm{const}$ mit $x \in [0, 1)$ erzeugt.

    Dazu wird zuerst die Verteilungsfunktion $F_X(x)$ von $p(x)$ durch integrieren in den Grenzen $[0, x]$ bestimmt.
    Damit die Wahrscheinlichkeiten $P(X < x) = F_X(x)$ nicht negativ sind, wird $\frac{1}{2}$ addiert:
    \begin{equation}
        F_X(x) = \frac{1}{2} + \frac{1}{\pi} \arctan(x)
    \end{equation}
    Umformen nach x ergibt die Transformation
    \begin{equation}
        x = \tan(\pi u - \frac{\pi}{2}) \;.
    \end{equation}

