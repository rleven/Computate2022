\setcounter{secnumdepth}{0}
\section{Aufgabe 0: Verständnisfragen}
\label{sec:auf0}

\subsection{1) \textbf{Was bedeutet \textit{numerische Stabilität} und wieso tritt sie auf?}}

Einen Algorithmus mit einer geringen Fehlerfortpflanzung bezeichnet man als numerisch stabil, einen solchen mit einer starken Fehlerfortpflanzung als numerisch instabil.
Eine Störung der Daten führt zu veränderten Ergebnissen.
Diese Störungsempfindlichkeit bezeichnet man auch als Kondition und sie stellt eine fundamentale Eigenschaft des mathematischen Zusammenhangs dar.

\subsection{2) \textbf{Wieso kann eine höhere \textit{Genauigkeit} (beispielsweise durch eine feine Diskretisierung) zu numerischer Instabilität führen?}}

Bei einer höheren Genauigkeit der eingegebenen Daten wird die Kondition des Algorithmus geändert, sodass es nicht mehr numerisch stabil ist.