\newpage
\section{Aufgabe2: Rundungsfehler}
\label{sec:auf2}

\subsection{a) für große x:}

Subtraktion (fast) gleich großer Zahlen ist hier das Problem.
\begin{align}
    \frac{1}{\sqrt{x}} - \frac{1}{\sqrt{x+1}} 
    = \frac{\sqrt{x+1}-\sqrt{x}}{\sqrt{x+1}\sqrt{x}} 
    = \frac{x}{\sqrt{x}\left(x+1\right)+\sqrt{x+1}x}
\end{align}


\subsection{b) für kleine x:}

$1-\cos(x)$ ist instabil für x nahe Null.
\begin{align}
    & \frac{1-\cos(x)}{\sin(x)}\\
    =& \frac{ 1 - \sqrt{1-\sin^2(x)} (1 + \sqrt{1-\sin^2(x)}) }{ 1 + \sqrt{1-\sin^2(x)} }\\
    =& \frac{ 1 - \left( 1-\sin^2(x) \right) }{ 1 + \cos(x)}\\
    =& \frac{ \sin^2(x) }{ 1 + \cos(x)}
\end{align}

\subsection{c) für kleine delta:}

Subtraktion (fast) gleich großer Zahlen ist hier das Problem.
\begin{align}
    \sin(x + \delta) - \sin(x)\\
    = 
\end{align}