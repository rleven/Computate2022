\newpage
\section{Aufgabe 1: Jacobi-Rotationen}
\label{sec:auf1}

Der Code zu dieser Aufgabe befindet sich in \textit{aufgabe1.py}.
\newline\\
Gegeben war die Matrix
\[
\begin{pmatrix}
    1 & -2 & 2 & 4\\
    -2 & 3 & -1 & 0\\
    2 & -1 & 6 & 3\\
    4 & 0 & 3 & 5
\end{pmatrix}
\]
, die mittels der Jacobi-Rotation in ihre Diagonalform mit Eigenwerten transformiert werden sollte.\\
Als Ergebnis dieser Jacobi-Rotation mit Abbruchbedingung $\sum_{i\neq j}|a_{ij}|^2 < 10^{-6}$ erhielten wir für die Eigenwerte:
\[
\begin{pmatrix}
    0.64807889 & 0 & 0 & \approx0\\
    0 & 3 & 0 & 0\\
    0 & 0 & 6 & 0\\
    \approx0 & 0 & 0 & 3.24039446
\end{pmatrix}
\]
Die mittels \textit{numpy.linalg.eig(A)} berechneten Eigenwerte der Anfangsmatrix sind:
\[
\begin{pmatrix}
    0.64807887 & 0 & 0 & 0\\
    0 & 3 & 0 & 0\\
    0 & 0 & 6 & 0\\
    0 & 0 & 0 & 3.24039448
\end{pmatrix}
\]
Die Unterschiede sind nur bei zwei Eigenwerten zu beobachten und betragen ca. \SI{0.0000006}{\percent} bis \SI{0.0000031}{\percent}.