\setcounter{secnumdepth}{0}
\section{Aufgabe 0: Verständnisfragen}
\label{sec:auf0}

\subsection{1) Welche Diagonalisierungsalgorithmen kennen Sie und von welcher Ordnung ist ihre Laufzeit?}
Die uns bekannten Diagonalisierungsalgorithmen sind:
\begin{itemize}
    \item Jacobi-Rotation
    \item Householder Algorithmus
    \item QR-Iteration
    \item Potenzmethode
\end{itemize}
Die Jacobi-Rotation hat eine Laufzeit von $23N^3$ bis $40N^3$ bis die Eigenwerte bestimmt sind.
\newline\\
Der Householder Algorithmus hat eine Laufzeit, die ähnlich zur Jacobi-Rotation ist, also $\mathcal{O}(N^3)$, jedoch ist der Vorfaktor hier wesentlich kleiner.
\newline\\
Die QR-Iteration hat eine Laufzeit von $\mathcal{O}(N^2)$.
\newline\\
Die Potenzmethode ist sehr praktikabel für sehr große $N$. Üblicherweise braucht sie eine Laufzeit von nur $\mathcal{O}(N)$.

\subsection{2) Warum ist die direkte Diagonalisierung größerer Systeme in der vollständigen Zustandsbasis oftmals nicht sinnvoll?}
Weil das zu viele Iterationen zur Folge haben könnte, sodass die Laufzeit zu hoch wäre.