\newpage
\section{Aufgabe 2: Fußball}
\label{sec:auf2}

\subsection{a)}

Hier kommt die Formel hin.

\subsection{b)}

Die Flugweite des Balls ist schematisch in \autoref{fig:flug} dargestellt.
\begin{figure}
    \centering
    \includegraphics[scale=0.7]{plots/distance.pdf}
    \caption{Kurve über die Distanz, die der Ball fliegt, bei unterschiedlichem Winkel.}
    \label{fig:flug}
\end{figure}

\newpage
\subsection{c)}

Der beste Abschusswinkel beträgt ca $0.713142°$.\\
Dafür sind die Bahnkurven in x-Richtung für 5 verschiedene Windstärken in \autoref{fig:bahn} dargestellt.

\begin{figure}
    \centering
    \includegraphics[scale=0.7]{plots/trajectory.pdf}
    \caption{X-Richtung Bahnkurven des Balls bei unterschiedlichem Wind.}
    \label{fig:bahn}
\end{figure}

\newpage
\subsection{d)}

Die Bereiche in denen der Ball im Tor landet sind $19.53° - 21.87°$ und $59.94° - 60.39°$.\\
Hier nochmal visuell dargestellt in \autoref{fig:winkel}.

\begin{figure}
    \centering
    \includegraphics[scale=0.7]{plots/goalangle.pdf}
    \caption{Winkel die das Tor treffen, einmal in Rad und einmal in Deg.}
    \label{fig:winkel}
\end{figure}

\subsection{f)}

Der Drehwinkel darf im Bereich $39.2$ bis $42.2$ liegen.