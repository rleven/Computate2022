\setcounter{secnumdepth}{0}
\section{Aufgabe 0: Verständnisfragen}
\label{sec:auf0}
\subsection*{a)}
    Eine Möglichkeit wäre es, in jedem Schritt das benutzte Verfahren für eine größere und eine kleinere Schrittweite durchzuführen.
    Die Differenz $\Delta y$ zwischen den beiden Werten wird dann als Fehlerabschätzung benutzt.
    Mithilfe der dieser Abschätzungen lässt sich nun eine Regel befolgen, die aus Herrn Kierfelds Skript stammt. \\[0.5cm]

    Wenn $\frac{|\Delta y|}{|\vec{y}|}$ kleiner einem gegebenen Fehler $\epsilon$ ist, dann verkleinere die Schrittweite $h$ zu $h' = h \left(\frac{\epsilon |\vec{y}|}{|\Delta y|}\right)^{1/n}$ und wiederhole den Schritt. \\[0.5cm]

    Wenn $\frac{|\Delta y|}{|\vec{y}|}$ größer einem gegebenen Fehler $\epsilon$ ist, dann vergrößere die Schrittweite $h$ zu $h' = h \left(\frac{\epsilon |\vec{y}|}{|\Delta y|}\right)^{1/n}$ und mache den nächsten Schritt.
    

\subsection*{b)}
    Die Schrittweite zu verkleinern ist generell sinnvoll, wenn die rechte Seite der DGL $\vec{f}$ sich nur schwach mit der Zeit ändert und die Schrittweite zu vergrößern ist generell sinnvoll, wenn die rechte Seite sich schnell ändert.

    Die Schrittweitenanpassung bei $\vec{f}$ zu machen, wo keiner der beiden Fälle so richtig zutrifft wäre einfach Verschwendung von Ressourcen, je nachdem, wie groß die Zeitskala gewählt wird.

    Aus diesem Grund wird in Teilaufgabe a) ein Verfahren beschrieben, welches bei jedem Schritt entscheidet, in welche Richtung die Schrittweite verändert werden soll.