\newpage
\section{Aufgabe 2: Anharmonischer Oszillator}
\label{sec:auf2}

\subsection{a)}
Es gilt für die stationäre Eigenwertgleichung:
\begin{equation}
    \hat{H} \ket{\psi(x)} = E \ket{\psi(x)}
\end{equation}
Also ausformuliert:
\begin{equation}
    \left(-\frac{\hbar^2}{2m}\partial^2_x + \frac{1}{2}m\omega^2x^2 + \lambda x^4\right)\psi(x) = E\psi(x)
    \label{eq:fullschro}
\end{equation}
Für $x = \alpha\xi$ ist die Ableitung dann:
\begin{equation}
    \partial_x = \alpha \partial_{\xi}
\end{equation}
Eingesetzt in \autoref{eq:fullschro} und zusätzlich etwas umgeformt ergibt sich:
\begin{equation}
    \left(-\partial_{\xi}^2 + \frac{m^2\omega^2}{\hbar^2}\xi^2 + \lambda\cdot \frac{2\alpha^2m}{\hbar^2} \xi^4\right) \psi(\xi) = \frac{2m\beta}{\alpha^2\hbar^2}\epsilon\ \psi(\xi)
    \label{eq:halfform}
\end{equation}
Daraus lassen sich bereits die Einheiten der Länge und Energien ablesen.\\
Sie beträgt:
\begin{equation}
    \left[\xi\right] = \frac{\hbar}{m\omega} \quad\text{und}\quad \left[\epsilon\right] = \frac{\alpha^2\hbar^2}{2m\beta}
\end{equation}
Damit wird die Gleichung zu:
\begin{equation}
    \left(-\partial_{\xi}^2 + \xi^2 + \lambda\cdot \frac{2\alpha^2\hbar^2}{m^3\omega^4} \xi^4\right) \psi(\xi) = \epsilon\ \psi(\xi)
    \label{eq:fullform}
\end{equation}
Aus \autoref{eq:fullform} ist die Relation von $\tilde{\lambda}$ zu $\lambda$ ablesbar:
\begin{equation}
    \tilde{\lambda} = \frac{2\alpha^2\hbar^2}{m^3\omega^4}\lambda
\end{equation}