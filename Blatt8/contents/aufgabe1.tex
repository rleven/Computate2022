\newpage
\section{Aufgabe 1: Profiling zur Untersuchung eines Algorithmus}
\label{sec:auf1}
In dieser Aufgabe sollte ein LGS mithilfe der vorgefertigten QR-Zerlegung im Modul \verb|Eigen| gelöst werden.

\subsection{a)}
    Dabei wurden die folgenden drei Schritte implementiert:
    \begin{itemize}
        \item[1)] Erstelle eine Matrix $M$ der Größe $N \times N$ mit \verb|Eigen::MatrixXd::Random|.
        \item[2)] Führe eine QR-Zerlegung mit \verb|Eigen| durch.
        \item[3)] Löse anschließend das LGS mit einer Methode aus \verb|Eigen|.
    \end{itemize}

\subsection{b)}
    \vspace{-0.5cm}
    \begin{figure}[H]
        \centering
        \includegraphics[width=0.48\textwidth]{plots/Laufzeiten.png} \vspace*{-0.5cm}
        \caption{Hier sind die jeweiligen Laufzeiten der drei Schritte in einzelnen Plots gegen die Matrixgröße $N$ aufgetragen.}
        \label{fig:Laufzeiten}
    \end{figure}
    \FloatBarrier

\subsection{c)}
    Hier sollte die Gesamtlaufzeit des Programms bis $N=1.000.000$ extrapoliert werden.
    Dies wurde mit der Funktion
    \begin{equation}
        f(x) = ax^3 + bx^2 + cx + d
    \end{equation}
    durchgeführt.
    Man erkennt, dass die Funktion die ersten 50 Werte nicht gut fittet, jedoch die anderen Werte einigermaßen gut extrapoliert.
    Demnach würde das Programm ca. $\SI{e7}{s}$ lang brauchen, was ca. 116 Tagen entspräche.
    \begin{figure}[H]
        \centering
        \includegraphics[width=0.6\textwidth]{plots/Gesamtlaufzeit.png} \vspace*{-0.5cm}
        \caption{Hier ist die Gesamtlaufzeit des Algorithmus gegen die Matrixgröße $N$ aufgetragen.}
        \label{fig:Gesamtlaufzeit}
    \end{figure}

    In \autoref{fig:Laufzeiten} ist zu erkennen, dass die QR-Zerlegung die meiste Laufzeit bis $N=1000$ in Anspruch nimmt und auch die am stärksten wachsende Steigung aufweist.
    Bei größeren Matrizen würde es also am meisten Sinn machen die QR-Zerlegung zu optimieren.

\subsection{d)}
    Neben der Laufzeit könnte auch das Abspeichern und Auslesen der Matrizen bei großen Matrizen zum Problem werden.