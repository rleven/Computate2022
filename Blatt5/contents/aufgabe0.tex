\setcounter{secnumdepth}{0}
\section{Aufgabe 0: Verständnisfragen}
\label{sec:auf0}

\subsection{1) Welche Verfahren zum Lösen einer zeitunabhängigen partiellen Differentialgleichung kennen Sie?}
Die uns bekannte Verfahren sind:
\begin{itemize}
    \item Schussverfahren
    \item Jacobi-Iteration
    \item Gauß-Seidel-Iteration
    \item Relaxationsverfahren
\end{itemize}
Zu den Vorteilen:\\
Das Schussverfahren zeichnet sich dadurch aus, dass die Ableitung einen sogenannten Schussparameter $\alpha$ als Steigung hat.
Dieser Parameter wird nun solange verändert, bis die Randbedingungen erfüllt sind.
\newline\\
Die Jacobi-Iteration 

\subsection{2) Wie unterscheidet sich das FTCS-Schema von dem Crank-Nicolson-Schema?}