\setcounter{secnumdepth}{0}
\newpage
\section{Aufgabe 1: Bifurkationsdiagramme}
\label{sec:auf1}
In dieser Aufgabe sollten Bifurkationsdiagramme für zwei Abbildungen geplottet werden.

Nach der Implementierung der Iterationsvorschrift der Abbildungen soll diese $N$-mal durchlaufen werden (bei uns 200-mal), sodass garantiert werden kann, dass ein Fixpunkt oder ein Orbit gefunden wurde. Dabei sollte das $r>0$ so gewählt werden, dass die $x$-Werte in jedem Schritt innerhalb ihrer jeweils vorgegebenen Intervalle bleiben.

Nach dem Warmlaufen werden dieselben Methoden benutzt um Fixpunkte und Orbits zu finden und als Punkte gegen den zugehörigen $r$-Wert im Bifurkationsdiagramm aufzutragen. Das $r$ wird dabei in Schritten von $\Delta r = 1 \cdot 10^{-3}$ variiert. Da von einem bestimmten Startwert $x_0$ nur manche Orbits erreicht werden können, werden bei uns jeweils zwei Startwerte gewählt, sodass ein komplettes Bifurkationsdiagramm entsteht.

Außerdem soll ein $r_{\infty}$ bestimmt werden, ab dem die Orbitsgröße höher als 64 ist und angenommen wird, dass das Bifurkationsdiagramm chaotisch wird. Zur Bestimmung der Orbitgröße wurde ein Wert $\epsilon = 10^{-4}$ eingeführt, der die Differenz zwischen zwei $x$-Werten angibt, ab der diese als gleich angesehen werden. Das $r_{\infty}$ hängt sehr stark von diesem Wert ab.

\subsection{a)}
    Die logistische Abbildung mit $x_n \in [0, 1]$ ist gegeben durch
    \begin{equation}
        x_{n+1} = r x_n (1 - x_n) \;.
    \end{equation}
    
    \begin{figure}[H]
        \centering
        \includegraphics[width=0.7\textwidth]{plots/fixpoints_logistic.png} \vspace{-0.3cm}
        \caption{Das Bifurkationsdiagramm zu der logistischen Abbildung im Intervall $r \in (0, 4)$.}
        \label{fig:fixpoints_logistic}
    \end{figure}
    \FloatBarrier
    
    Der maximale Wert, den der Parameter annehmen kann ist $r=4$, sonst würde das Maximum nicht mehr in dem Intervall liegen.
    
    Hier ist $r_{\infty} \approx 3,572$, also nah an der Feigenbaumkonstante von $3,570 \ldots$.

\subsection{b)}
    Die kubische Abbildung mit $x_n \in [-\sqrt{1+r}, \sqrt{1+r}]$ ist gegeben durch
    \begin{equation}
        x_{n+1} = r x_n - x_n^3 \;.
    \end{equation}
    Der maximale Wert, den der Parameter annehmen kann ist $r=3$, sonst würden die beiden Extrempunkte nicht mehr in dem Intervall liegen.

        \begin{figure}[H]
            \centering
            \includegraphics[width=0.7\textwidth]{plots/fixpoints_cubic.png} \vspace{-0.3cm}
            \caption{Das Bifurkationsdiagramm zu der kubischen Abbildung im Intervall $r \in (0, 3)$.}
            \label{fig:fixpoints_cubic}
        \end{figure}
        \FloatBarrier

        Hier ist $r_{\infty} \approx 1,006$.