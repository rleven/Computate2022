\newpage
\section{Aufgabe 1: Poisson-Gleichung}
\label{sec:auf1}
Hier basiert die numerische Lösung der Poisson-Gleichung auf dem Gauß-Seidel-Verfahren.
In den Teilaufgaben wird diese für verschiedene Randbedingungen für ein Gittersystem mit den Ausmaßen $Q = [0, 1]^2$ und der Diskretisierung $\Delta = 0.05$ bestimmt. D.h. es gibt 21 Gitterplätze.
Dabei soll das iterative Gauß-Seidel-Verfahren bei einer Fehlerschranke von $\kappa = 10^{-5}$ abgebrochen werden.

Im folgenden sind die sich zu den auf dem Übungsblatt genannten Rand- und Anfangsbedingungen (RB und AB) ergebenden Plots des Potentials $\Phi(x, y)$ dargestellt.

(Wir haben das elektrische Feld nicht berechnet, da wir nicht wussten in welche Richtung es zeigen soll und wie wir das darstellen sollten.)

\subsection{a)}
    \vspace*{-2cm}
    \begin{figure}[H]
        \centering
        \includegraphics[width=0.65\textwidth]{plots/Potential_a.png} \vspace*{-1cm}
        \caption{Das Potential wenn als AB im Inneren $\Phi(x,y)=1$ und als Dirichlet-RB außen $\Phi=0$ gilt. Es gibt keine Ladungen.}
        \label{fig:Potential_a}
    \end{figure}
    \FloatBarrier

\subsection{b)}
    Für diese Teilaufgabe gilt die Randbedingung, dass alle Gitterpunkte an der Seite von $Q$ mit $y=1$ ein Potential $\Phi(x, y=1) = 1$ besitzen und alle anderen Gitterpunkte nicht.

    Das numerische Ergebnis soll daraufhin mit dem analytischen Ergebnis verglichen werden.
    Es ist zu erkennen, dass die beiden Ergebnisse in ihrer Form ähneln, wobei bei der numerischen Lösung das Potential etwas homogener an der Seite verteilt ist.

    \begin{figure}[H]
        \centering
        \includegraphics[width=0.65\textwidth]{plots/Potential_b.png} \vspace*{-0.7cm}
        \caption{Das Potential für die oben genannten RB und AB. Es gibt keine Ladungen.}
        \label{fig:Potential_b}
    \end{figure}
    \FloatBarrier
    \vspace*{-1cm}
    \begin{figure}[H]
        \centering
        \includegraphics[width=0.65\textwidth]{plots/Potential_b_anal.png} \vspace*{-0.7cm}
        \caption{Das analytisch berechnete Potential mit den oben genannten RB und AB. Es gibt keine Ladungen.}
        \label{fig:Potential_b_anal}
    \end{figure}
    \FloatBarrier

\subsection{c)}
    \vspace*{-1.5cm}
    \begin{figure}[H]
        \centering
        \includegraphics[width=0.65\textwidth]{plots/Potential_c.png} \vspace*{-0.7cm}
        \caption{Das Potential wenn eine Ladung in die Mitte von $Q$ gesetzt wird. Als AB gilt überall $\Phi(x,y) = 0$}
        \label{fig:Potential_c}
    \end{figure}
    \FloatBarrier

\subsection{d)}
    \vspace*{-1.5cm}
    \begin{figure}[H]
        \centering
        \includegraphics[width=0.65\textwidth]{plots/Potential_d.png} \vspace*{-0.7cm}
        \caption{Das Potential zwei positive Ladungen auf die Gitterplätze $(5,5)^\top$, $(15,15)^\top$ und zwei negative Ladungen auf die Gitterplätze $(5,15)^\top$, $(15,5)^\top$ gesetzt werden.}
        \label{fig:Potential_d}
    \end{figure}
    \FloatBarrier

