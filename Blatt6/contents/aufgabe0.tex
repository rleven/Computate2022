\setcounter{secnumdepth}{0}
\section{Aufgabe 0: Verständnisfragen}
\label{sec:auf0}

\subsection{1) Welche Verfahren zum Lösen einer zeitunabhängigen partiellen Differentialgleichung kennen Sie?}
Die uns bekannte Verfahren sind:
\begin{itemize}
    \item Schussverfahren
    \item Jacobi-Iteration
    \item Gauß-Seidel-Iteration
    \item Relaxationsverfahren
\end{itemize}
Zu den Vorteilen:\\
Das Schussverfahren zeichnet sich dadurch aus, dass die Ableitung einen sogenannten Schussparameter $\alpha$ als Steigung hat.
Dieser Parameter wird nun solange verändert, bis die Randbedingungen erfüllt sind.
\newline\\
Die Jacobi-Iteration hat den Vorteil, dass sie immer konvergiert.
Sie ist die iterative Lösung einer Fixpunktgleichung.
\newline\\
Die Gauß-Seidel-Iteration ist ähnlich zur Jacobi-Iteration, der Unterschied hierzu ist, dass statt der vorherigen Iteration, die nächste Iteration auf der rechten Seite der Gleichung verwendet wird.
Der Vorteil ist der gleiche wie bei Jacobi, es konvergiert immer.
\newline\\
Das Relaxationsverfahren ist eine Kombination aus Jacobi und Gauß-Seidel, wobei noch die vorherige Lösung hinzugefügt wird.
\subsection{2) Wie unterscheidet sich das FTCS-Schema von dem Crank-Nicolson-Schema?}
Das FTCS-Schema, also "Forward in Time, Centered in Space", nimmt die umliegenden Werte, um den zentrieren Wert für einen Zeitschritt zu berechnen.
Das Crank-Nicolson-Schema ist auch stabil für große $\Delta t$ und nimmt zusätzlich auch die nächsten Nachbarn des nächsten Zeitschritts.