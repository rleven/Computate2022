\newpage
\section{Aufgabe 2: Der Wetterfrosch}
\label{sec:auf2}

\subsection{a)}
In der Datei \textit{aufgabe2.py} wurde das Runge-Kutta Verfahren vierter Ordnung implementiert.\\
Als Anfangsbedingungen haben wir uns für vier Bahnen entschieden:
\begin{equation}
    \left(\begin{array}{c}0\\0\\0\end{array}\right) \ \text{,} \ \left(\begin{array}{c}1\\1\\1\end{array}\right) \ \text{,} \ \left(\begin{array}{c}0.1\\0.1\\0.1\end{array}\right) \ \text{und} \ \left(\begin{array}{c}5\\10\\15\end{array}\right)
\end{equation}
Diese wurden jeweils mit dem Parameter $r=20$ und $r=28$ berechnet und graphisch wie folgt dargestellt.

\subsection{b)}
\textbf{In den folgenden Abbildungen ist die horizontale Achse immer die x-Achse und die vertikale Achse immer die y-Achse.}\\
Zunächst wurde der Verlauf der Koordinaten auf der xy-Ebene dargestellt.\\
In \autoref{fig:xy} sind die Verläufe für alle Anfangsbedingungen dargestellt, bis auf die Anfangsbedingung $\left(0,0,0\right)^T$.
Diese wurde in keiner Grafik dargestellt, da hierbei die Differentialgleichungen nach Runge-Kutta wieder nur $\left(0,0,0\right)^T$ ergeben.
\begin{figure}[H]
    \centering
    \includegraphics[angle=-90,scale=0.65]{plots/2Dlorenz.pdf}
    \caption{Hier sind die sechs Verläufe in der xy-Ebene für verschiedene Startbedingungen dargestellt.}
    \label{fig:xy}
\end{figure}
Die Pointcaré-Schnitte sind in \autoref{fig:care} dargestellt.
\begin{figure}[H]
    \centering
    \includegraphics[angle=-90,scale=0.65]{plots/pointcare.pdf}
    \caption{Hier sind die sechs verschiedenen Pointcaré-Schnitte für z=20 dargestellt.}
    \label{fig:care}
\end{figure}
Und zuletzt wurden alle Verläufe in drei Dimensionen in \autoref{fig:dreid} dargestellt.
\begin{figure}[H]
    \centering
    \includegraphics[angle=-90,scale=0.65]{plots/3Dlorenz.pdf}
    \caption{Hier sind die sechs Verläufe in drei Dimensionen für verschiedene Startbedingungen dargestellt.}
    \label{fig:dreid}
\end{figure}