\newpage
\section{Aufgabe 3: Eindimensionale Integrale}
\label{sec:auf3}

Bei dieser Aufgabe sollten zwei Integrale numerisch mit Trapezregel, Mittelpunktsregel und Simpsonregel berechnet werden.

\subsection{a)}
Das gegebene Integral mit der Form
\begin{equation}
    I_1 = \int_1^{100}\frac{exp(-x)}{x}dx
    \label{eq:auf3a}
\end{equation}
wurde jeweils für unterschiedliche Intervallbreiten $h$, beziehungsweise unterschiedlicher Anzahl an Intervallen $N$, mittels der drei Regeln ausgewertet.\\
Die Ergebnisse sind dabei in den Textdateien trapez\_a.txt, midpoint\_a.txt und simpson\_a.txt gespeichert worden.
Wie in der Aufgabenstellung angegeben, wurden die Intervallbreiten halbiert bis ein relativer Fehler von $<10^{-4}$ zur vorherigen Iteration bestand.\\
In den cpp-Funktionen wurde die Anzahl der Intervalle $N$ übergeben, sodass diese zur Halbierung von $h$ verdoppelt werden mussten.\\
Hieraus ergab sich der Plot aus \autoref{fig:aufgabe3_a}.
\begin{figure}
    \centering
    \includegraphics[scale=0.7]{plots/aufgabe3a.pdf}
    \caption{In diesem Graphen sind die Verläufe der drei numerischen Regeln für die Integration \ref{eq:auf3a} dargestellt.
    Es ist zu sehen, wie die Trapezregel und Simpsonregel ähnlich schnell ansteigen, die Mittelpunktsregel jedoch mit größeren Intervallsbreiten das genauere Ergebnis produziert.}
    \label{fig:aufgabe3_a}
\end{figure}
Die Trapez- und Simpsonregel sind erkennbar im selben Bereich und nähern sich dem analytischen Ergebnis langsamer als die Mittelpunktsregel.

\subsection{b)}
Das gegebene Integral der Form
\begin{equation}
    I_2 = \int_0^1 x sin(\frac{1}{x})dx
    \label{eq:auf3b}
\end{equation}
wurde wie im Aufgabenteil a) numerisch berechnet.\\
Auch hier befinden sich die Ergebnisse in den Textdateien trapez\_b.txt, midpoint\_b.txt und simpson\_b.txt.
Aus diesen Daten ergab sich der Plot aus \autoref{fig:aufgabe3_b}.
\begin{figure}
    \centering
    \includegraphics[scale=0.7]{plots/aufgabe3b.pdf}
    \caption{In diesem Graphen sind die Verläufe der drei numerischen Regeln für die Integration \ref{eq:auf3b} dargestellt.
    Es ist zu sehen, wie die Trapezregel sowie Mittelpunktsregel und Simpsonregel ähnlich schnell gegen den analytischen Wert gehen. Nur bei hohen Intervallbreiten gibt es deutliche Unterschiede.}
    \label{fig:aufgabe3_b}
\end{figure}
Laut dem Graphen konvergieren alle drei Regeln ähnlich schnell zum analytischen Wert.\\
Äquivalent zu Aufgabenteil a) wurde auch hier die Intervallbreite bis zu einem relativen Fehler von $<10^{-4}$ halbiert.