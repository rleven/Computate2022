\setcounter{secnumdepth}{0}
\section{Aufgabe 0: Verständnisfragen}
\label{sec:auf0}

\subsection{1) \textbf{Welches der in Aufgabe 2 und 3 genannten Integrationsverfahren hat die höchste Genauigkeit? Welches ist am besten?}}
Die höchste Genauigkeit hat die Simpsonregel mit $\mathcal{O}(N^{-4})$, weil die Simpsonregel zusätzlich zur ersten Ableitung in der Taylorentwicklung auch die zweite Ableitung berücksichtigt.
Im Gegensatz dazu hat Trapezregel  und Mittelpunktsregel je $\mathcal{O}(N^{-2})$ Fehler, da bei beiden nur die erste Ableitung betrachtet wird.\\
Das beste Verfahren ist je nach gegebenem Integral zu wählen.
So wird die Trapezregel für \textit{lineare} Funktionen exakt.
Die Mittelpunktsregel ist von Vorteil, wenn die Funktion an den Rändern der Integration problematisch ist (z.B. Teilung durch 0).
Im Vergleich zur Trapezregel muss also bei der Mittelpunktsregel nicht exakt an den Intervallrändern ausgewertet werden.
Die Simpsonregel ist am besten, wenn die Funktion quadratisch ist, denn dann wird die Regel exakt.

\subsection{2) \textbf{Welchen Grad darf ein Polynom maximal haben, um mit der Simpsonregel exakt integrierbar zu sein?}}
Die Simpsonregel wird exakt für quadratische Funktionen, also wäre der maximale Grad $2$.