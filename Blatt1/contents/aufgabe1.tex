\newpage
\section{Aufgabe 1: Numerische Differentiation}
  \label{sec:auf1}
  Die numerische Ableitung mit Hilfe der Zweipunkt- und der Vierpunktregel
  \begin{align}
    f'_{\mathrm{Zweipunkt}}(x,h) &= \frac{f(x+h) - f(x-h)}{2h} \\[10pt]
    f'_{\mathrm{Vierpunkt}}(x,h) &= \frac{-f(x+2h) + 8f(x+h) - 8f(x-h) - f(x-2h)}{12h}
  \end{align}
  soll anhand der folgenden beiden Funktionen untersucht werden.
  \begin{align}
    \label{eqn:aufgabe1funktion1}
    f_1(x) &= \sin(x)\\[2pt]
    \label{eqn:aufgabe1funktion2}
    f_2(x) &= \left\{
      \begin{matrix}
        2\cdot \lfloor\frac{x}{\pi} \rfloor -\cos(x \;\;\text{mod}\;\; \pi) + 1 \quad \text{für} \quad x \geq 0 \\[5pt]
        2\cdot \lfloor\frac{x}{\pi} \rfloor +\cos(x \;\;\text{mod}\;\; \pi) + 1 \quad \text{für} \quad x < 0
      \end{matrix}
      \right.
  \end{align}

  \subsection{a)}
    In dieser Teilaufgabe sollte die Zweipunktregel an der Funktion \ref{eqn:aufgabe1funktion1} untersucht werden.
    Dazu wurde die Ableitung $f'_1(x_1)$ mithilfe der Zweipunktregel an einer willkürlich ausgesuchten Stelle $x_1 = \pi$ berechnet, wobei die Schrittweite $h$ von $10^{-3}$ bis $10^{-1}$ variiert wurde.
    In \autoref{fig:aufgabe1_a1} ist die numerisch bestimmte Ableitung $f'_{\mathrm{1,Zweipunkt}}(\pi)$ als auch der tatsächliche Wert der Ableitung $f'_1(\pi) = -1$ gegen die Schrittweite aufgetragen, sodass die beiden Werte verglichen werden können.
    
    \begin{figure}[ht]
      \center
      \includegraphics[width=0.65\textwidth]{plots/aufgabe1_a1.pdf}
      \caption{Die numerische und tatsächliche Wert der Ableitung $f'_1(\pi)$ sind hier dargestellt und gegen $h$ aufgetragen.}
      \label{fig:aufgabe1_a1}
    \end{figure}

    Links im Graphen ist der Fehler durch die Auslöschung bei zu kleinem $h$ und rechts der Abbruchfehler bei zu großem $h$ zu erkennen.
    Für die weiterführenden Rechnungen wird $h = 10^{-2}$ gewählt.

    In \autoref{fig:aufgabe1_a2} wird der absolute Fehler, der mit der Zweipunktregel bestimmten, numerischen von der analytischen Ableitung $\Delta f'_{\mathrm{1,Zweipunkt}}(x) = f'_1(x) - f'_{\mathrm{1,Zweipunkt}}(x)$ im Intervall $x \in [-\pi, \pi]$ dargestellt.
    Es ist zu erkennen, dass der Fehler der Form eines Cosinus folgt.

    \begin{figure}[ht]
      \center
      \includegraphics[width=0.65\textwidth]{plots/aufgabe1_a2.pdf}
      \caption{Der Fehler zwischen der mit der Zweipunktregel bestimmten numerischen und tatsächlichen Ableitung $f'_1(\pi)$ sind hier dargestellt und im Intervall $[-\pi, \pi]$ gegen $x$ aufgetragen.}
      \label{fig:aufgabe1_a2}
    \end{figure}
