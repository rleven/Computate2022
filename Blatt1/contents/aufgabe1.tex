\newpage
\section{Aufgabe 1: Numerische Differentiation}
  \label{sec:auf1}
  Die numerische Ableitung mit Hilfe der Zweipunkt- und der Vierpunktregel
  \begin{align}
    f'_{\mathrm{Zweipunkt}}(x,h) &= \frac{f(x+h) - f(x-h)}{2h} \\[10pt]
    f'_{\mathrm{Vierpunkt}}(x,h) &= \frac{-f(x+2h) + 8f(x+h) - 8f(x-h) - f(x-2h)}{12h}
  \end{align}
  soll anhand der folgenden beiden Funktionen untersucht werden.
  \begin{align}
    \label{eqn:aufgabe1funktion1}
    f_1(x) &= \sin(x)\\[2pt]
    \label{eqn:aufgabe1funktion2}
    f_2(x) &= \left\{
      \begin{matrix}
        2\cdot \lfloor\frac{x}{\pi} \rfloor -\cos(x \;\;\text{mod}\;\; \pi) + 1 \quad \text{für} \quad x \geq 0 \\[5pt]
        2\cdot \lfloor\frac{x}{\pi} \rfloor +\cos(x \;\;\text{mod}\;\; \pi) + 1 \quad \text{für} \quad x < 0
      \end{matrix}
      \right.
  \end{align}

  \subsection{a)}
    In dieser Teilaufgabe sollte die Zweipunktregel an der Funktion \ref{eqn:aufgabe1funktion1} untersucht werden.
    Dazu wurde die Ableitung $f'_1(x_1)$ mithilfe der Zweipunktregel an einer willkürlich ausgesuchten Stelle $x_1 = \pi$ berechnet, wobei die Schrittweite $h$ von $10^{-3}$ bis $10^{-1}$ variiert wurde.
    In \autoref{fig:aufgabe1_a1} ist die numerisch bestimmte Ableitung $f'_{\mathrm{1,Zweipunkt}}(\pi)$ als auch der tatsächliche Wert der Ableitung $f'_1(\pi) = -1$ gegen die Schrittweite aufgetragen, sodass die beiden Werte verglichen und die Schrittweite, wo sie sich am nächsten sind abgelesen werden kann.
    
    \begin{figure}[ht]
      \centering
      \includegraphics[width=0.65\textwidth]{plots/aufgabe1_a1.pdf} \vspace*{-0.6cm}
      \caption{Der numerische und tatsächliche Wert der Ableitung $f'_1(\pi)$ sind hier dargestellt und gegen $h$ aufgetragen.}
      \label{fig:aufgabe1_a1}
    \end{figure}
    \FloatBarrier

    Links im Graphen ist der Fehler durch die Auslöschung bei zu kleinem $h$ und rechts der Abbruchfehler bei zu großem $h$ zu erkennen.
    Für die weiterführenden Rechnungen wird $h = 10^{-2}$ gewählt.

    In \autoref{fig:aufgabe1_a2} wird der absolute Fehler, der mit der Zweipunktregel bestimmten, numerischen von der analytischen Ableitung $\Delta f'_{\mathrm{1,Zweipunkt}}(x) = f'_1(x) - f'_{\mathrm{1,Zweipunkt}}(x)$ im Intervall $x \in [-\pi, \pi]$ dargestellt.
    Es ist zu erkennen, dass der Fehler der Form der Ableitung also einem Cosinus gleicht.

    \begin{figure}[ht]
      \centering
      \includegraphics[width=0.65\textwidth]{plots/aufgabe1_a2.pdf} \vspace*{-0.6cm}
      \caption{Der Fehler zwischen der mit der Zweipunktregel bestimmten numerischen und tatsächlichen Ableitung $f'_1(x) = \cos(x)$ ist hier dargestellt und im Intervall $[-\pi, \pi]$ gegen $x$ aufgetragen.}
      \label{fig:aufgabe1_a2}
    \end{figure}
    \FloatBarrier

  \subsection{b)}
    Hier wurde die doppelte Zweipunktregel zur Bestimmung der zweiten Ableitung an der Funktion \ref{eqn:aufgabe1funktion1} untersucht.
    \begin{equation}
      f''_{\mathrm{Zweipunkt}}(x) = \frac{f(x+h) - 2f(x) + f(x-h)}{h^2}
    \end{equation}
    Es wurde analog zu Aufgabenteil a) verfahren. \\
    Die zweite Ableitung $f''_1(x_1)$ wurde wieder an einer willkürlich ausgesuchten Stelle $x_1 = \pi / 2$ berechnet, wobei die Schrittweite $h$ von $10^{-3}$ bis $1$ variiert wurde.
    In \autoref{fig:aufgabe1_b1} wurde die Schrittweite $h = 7 \cdot 10^{-2}$ per Augenmaß abgelesen.
    
    \begin{figure}[H]
      \centering
      \includegraphics[width=0.65\textwidth]{plots/aufgabe1_b1.pdf} \vspace*{-0.6cm}
      \caption{Der numerische und tatsächliche Wert der zweiten Ableitung $f''_1(\pi/2)$ sind hier dargestellt und gegen $h$ aufgetragen.}
      \label{fig:aufgabe1_b1}
    \end{figure}
    \FloatBarrier

    In \autoref{fig:aufgabe1_b2} wird der absolute Fehler von der analytischen Ableitung $\Delta f'_{\mathrm{1,Zweipunkt}}(x) = f''_1(x) - f''_{\mathrm{1,Zweipunkt}}(x)$ im Intervall $x \in [-\pi, \pi]$ dargestellt. \\
    Auch hier ist deutlich zu erkennen, dass der Fehler der Form der zweiten Ableitung $f''_1(x) = -\sin(x)$ entspricht.

    \begin{figure}[H]
      \centering
      \includegraphics[width=0.65\textwidth]{plots/aufgabe1_b2.pdf} \vspace*{-0.6cm}
      \caption{Der Fehler zwischen der numerisch und analytisch bestimmten zweiten Ableitung ist hier dargestellt und im Intervall $[-\pi, \pi]$ gegen $x$ aufgetragen.}
      \label{fig:aufgabe1_b2}
    \end{figure}
    \FloatBarrier

    Die Fehler der ersten und zweiten Ableitung besitzen beide die Formen der jeweiligen Ableitung.

  \subsection{c)}
    \begin{figure}[H]
      \centering
      \includegraphics[width=0.65\textwidth]{plots/aufgabe1_c1.pdf} \vspace*{-0.6cm}
      \caption{Der numerische und tatsächliche Wert der Ableitung $f'_1(\pi)$ sind hier dargestellt und gegen $h$ aufgetragen.}
      \label{fig:aufgabe1_c1}
    \end{figure}
    \FloatBarrier

    \begin{figure}[H]
      \centering
      \includegraphics[width=0.65\textwidth]{plots/aufgabe1_c2.pdf} \vspace*{-0.6cm}
      \caption{Der Fehler zwischen der numerisch und analytisch bestimmten Ableitung ist hier dargestellt und im Intervall $[-\pi, \pi]$ gegen $x$ aufgetragen.}
      \label{fig:aufgabe1_c2}
    \end{figure}
    \FloatBarrier
    
    In \autoref{fig:aufgabe1_c2} wurde die Schrittweite $h = 0,15$ gewählt, sodass der Fehler in der selben Größenordnung liegt wie in \autoref{fig:aufgabe1_a2} für die Zweipunktregel.
    Das zeigt, dass die Vierpunktregel erst bei viel größeren Schrittweiten einen genauso großen Fehler aufweist wie die Zweipunktregel.
    Für dieses Beispiel ist also die Vierpunktregel genauer.

  \subsection{d)}
    Nun sollte der selbe Vergleich zwischen Zwei- und Vierpunktregel an der 2. Funktion durchgeführt werden, wie schon in Teilaufgabe a) und c).
    Die Ableitung von \ref{eqn:aufgabe1funktion2} ist gegeben als
    \begin{equation}
      f_2'(x) = \left\{
        \begin{matrix}
          \sin(x \;\;\text{mod}\;\; \pi) \quad \text{für} \quad x \geq 0 \\[2pt]
          -\sin(x \;\;\text{mod}\;\; \pi) \quad \text{für} \quad x < 0
        \end{matrix}
      \right.
    \end{equation}
    
    \begin{figure}[ht]
      \centering
      \begin{subfigure}{0.49\textwidth}
          \includegraphics[width=\textwidth]{plots/aufgabe1_d1.pdf} \vspace*{-0.6cm}
          \caption{Auswahl der Schrittweite $h$ für die Zweipunktregel.}
          \label{fig:aufgabe1_d1}
      \end{subfigure}
      \hfill
      \begin{subfigure}{0.49\textwidth}
          \includegraphics[width=\textwidth]{plots/aufgabe1_d3.pdf} \vspace*{-0.6cm}
          \caption{Auswahl der Schrittweite $h$ für die Vierpunktregel.}
          \label{fig:aufgabe1_d3}
      \end{subfigure}
      
  \end{figure}
  
  \FloatBarrier

  \begin{figure}[ht]
    \centering
    \begin{subfigure}{0.49\textwidth}
        \includegraphics[width=\textwidth]{plots/aufgabe1_d2.pdf} \vspace*{-0.6cm}
        \caption{Fehler für die Zweipunktregel.}
        \label{fig:aufgabe1_d2}
    \end{subfigure}
    \hfill
    \begin{subfigure}{0.49\textwidth}
        \includegraphics[width=\textwidth]{plots/aufgabe1_d3.pdf} \vspace*{-0.6cm}
        \caption{Fehler für die Vierpunktregel.}
        \label{fig:aufgabe1_d4}
    \end{subfigure}
    
  \end{figure}

\FloatBarrier